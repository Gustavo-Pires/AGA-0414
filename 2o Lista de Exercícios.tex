\documentclass[a4paper, 12pt]{article}
\usepackage{graphicx}
\usepackage{amsmath}
\usepackage{afterpage}
\usepackage[utf8]{inputenc}
\usepackage[portuguese]{babel}
\usepackage{fancyhdr}
\usepackage{lipsum} 
\usepackage{caption} 

% Configurações da página
\pagestyle{fancy}
\fancyhf{}
\rhead{Página \thepage}
\renewcommand{\footrulewidth}{0.5pt} % Adiciona uma linha no rodapé
\lfoot{IAG - Universidade de São Paulo  }
\rfoot{Gustavo Pires Bertaco}

% Título do documento
\title{ \textbf{Pré Prova- 2ª Lista de exercícios} \\ AGA0414 - Métodos Observacionais em Astrofísica I (2023)}
\author{Gustavo Pires Bertaco, NUSP: 12745240}
\date{\today}


\begin{document}

\maketitle
\section*{\large Questão 1}
\noindent \textbf{Explique o que é um bias, como ele é obtido e para que é utilizado.} \\
\\ 
\textbf{Bias:} É uma imagem obtida com tempo de exposição zero. O "bias" é a leitura de um CCD sem incidência de fótons. Ele é usado para estimar o valor real de zero do CCD.

O bias é obtido ao fechar a tampa do telescópio ou câmera, garantindo que nenhuma luz atinja o sensor. Esta leitura é essencial para correções nas imagens astronômicas, permitindo que os cientistas subtraiam o sinal do bias das imagens obtidas com exposições mais longas, revelando assim detalhes astronômicos mais precisos e reduzindo o ruído nas observações.

\section*{Questão 2}
\noindent \textbf{Explique o que é um flat-field, como ele é obtido e para que é utilizado.}

\textbf{FLAT:} Uma imagem de uma superfície uniformemente iluminada. O flat-field é obtido expondo o sensor a uma fonte de luz homogênea, geralmente uma fonte difusa como o céu ao anoitecer ou ao amanhecer, ou um painel iluminado de forma uniforme. Essa imagem é usada para estimar e corrigir as diferenças de sensibilidade em todo o CCD, bem como as variações de ganho pixel a pixel.

O flat-field é essencial na correção de imperfeições ópticas e eletrônicas do detector, como manchas e padrões de sensibilidade desigual. Após adquirir uma imagem de flat-field, ela é utilizada para corrigir as imagens astronômicas, garantindo uma representação mais precisa e uniforme das características dos objetos observados.


\section*{Questão 3}
\noindent \textbf{Durante um turno de observação, um astrônomo precisa determinar o seeing em que um objeto foi observado. Para isso ele obtém o perfil radial de uma estrela de campo, que é mostrado na figura abaixo. Considerando que a escala de placa do CCD é 0,18”/pxl, qual foi o seeing obtido?} \\

O ideal é repetir esse procedimento para pelo menos 3
observações (começo,meio e final da noite).
Usando o usando o imexamine é possivel estimar o FWHM(Full Width at Half Maximum) para algumas estrelas de campo, que posteriormente é usado para determinar o valor do seeing utilizando a relação entre o FWHM  da estrela e a escala de placa do CCD. O FWHM representa a largura da estrela no ponto em que sua intensidade luminosa é metade do valor máximo.

Dado que o FWHM é 7.25 pixels e a escala de placa do CCD é 0.18" por pixel, você pode calcular o seeing da seguinte forma:

\[
\text{Seeing} = \text{FWHM} \times \text{Escala de Placa do CCD}
\]

Substituindo os valores dados:

\[
\text{Seeing} = 7.25 \, \text{pixels} \times 0.18" /\text{pixel}
\]

\[
\text{Seeing} = 1.305"
\]

Portanto, o seeing obtido durante a observação do objeto foi de $1.305"$.


\section*{Questão 4}
\noindent \textbf{Considerando que estamos usando o Telescópio 1,6 m Perkin-Elmer do OPD/LNA com a câmera iXon - 4269, com dimensão de 1024 X 1024 e tamanho de pixel de 13,5 µm, instalada no foco Cassegrain com razão focal f/10, responda} \\
\begin{itemize}
    \item \textbf{A)} Qual é a escala de placa para o telescópio nessa configuração? Expresse em unidades de arcmin/pixel e arcmin/mm
    \item \textbf{B)}Qual o campo de visada útil com a câmera Ixon? Expresse em arcminutos e arcsegundos.
    \item \textbf{C)}Considerando uma possível configuração do telescópio que reduza a razão focal para F/5, qual seria a escala de placa e campo de visada útil com a câmera Ixon?
    \item \textbf{D)}Considerando o item c), qual o parâmetro óptico que é modificado ao selecionar que a razão focal do telescópio seja reduzida? Como isso, na prática, é realizado?
\end{itemize}

\section*{Questão 5}
\noindent \textbf{A figura retratada exibe o aglomerado estelar aberto Messier 67, situado na constelação de Câncer, capturado num campo de visão de 15' x 15'. Descreva o fenômeno observado na imagem, oferecendo evidências que sustentem sua interpretação e, se necessário, discuta possíveis métodos de correção.} \\

\section*{Questão 6}
\noindent \textbf{A imagem abaixo é de uma estrela. Descreva o efeito observado na imagem, sua origem e indique possíveis correções.} \\


O efeito observado na imagem é o blooming.

O blooming ocorre quando a carga em um pixel excede o nível de saturação e a carga começa a se espalhar para pixels adjacentes.

Normalmente, os sensores CCD são projetados para permitir o deslocamento vertical fácil da carga, mas barreiras potenciais são criadas para reduzir o fluxo para os pixels horizontais.

Portanto, o excesso de carga fluirá preferencialmente para os vizinhos verticais mais próximos. O blooming, portanto, produz uma característica linha vertical


\section*{Lista Anterior}
\subsection{Telescópios}
\textbf{Vantagens do Uso de Espelhos:}
\begin{itemize}
    \item \textbf{Aberração Cromática Reduzida:} Os espelhos não sofrem de aberração cromática, um problema comum em lentes que distorce as cores e reduz a qualidade da imagem.
    \item \textbf{Custo Mais Baixo:} Geralmente, espelhos podem ser produzidos a um custo mais baixo do que lentes de alta qualidade, especialmente para grandes diâmetros.
    \item \textbf{Design Óptico Simplificado:} Os sistemas ópticos com espelhos podem ter um design mais simples e compacto, facilitando a construção de telescópios de grande abertura.
    \item \textbf{Menor Manutenção:} Espelhos geralmente requerem menos manutenção do que lentes, pois não estão expostos ao ambiente externo e não estão sujeitos a arranhões ou sujeira com tanta facilidade.
\end{itemize}

\textbf{Desvantagens do Uso de Espelhos:}
\begin{itemize}
    \item \textbf{Requere Superfícies Mais Precisas:} Espelhos precisam ter superfícies extremamente precisas para evitar distorções na imagem, o que pode aumentar o custo do processo de fabricação.
    \item \textbf{Problemas com Reflexos e Difração:} Espelhos podem gerar reflexos e difração que podem interferir com a qualidade da imagem, especialmente em telescópios de alta potência.
\end{itemize}

\textbf{Vantagens do Uso de Lentes:}
\begin{itemize}
    \item \textbf{Ampla Gama de Materiais:} Lentes podem ser fabricadas a partir de uma variedade de materiais, permitindo a correção de diversas propriedades ópticas.
    \item \textbf{Menor Sensibilidade ao Ambiente:} Lentes são menos sensíveis a mudanças ambientais, como variações de temperatura, do que os espelhos.
\end{itemize}

\textbf{Desvantagens do Uso de Lentes:}
\begin{itemize}
    \item \textbf{Aberração Cromática:} Lentes sofrem de aberração cromática, que pode distorcer as cores na imagem e requer correção adicional.
    \item \textbf{Custo Mais Elevado para Alta Qualidade:} Lentes de alta qualidade podem ser caras, especialmente para grandes diâmetros e correção de aberrações.
\end{itemize}

\textbf{Telescópios Refratores:}
\begin{itemize}
    \item \textbf{Abordagem com Lentes Acromáticas:} Telescópios refratores utilizam lentes acromáticas, que são compostas por dois ou mais tipos de vidro com índices de refração diferentes. Essas lentes são projetadas para minimizar a aberração cromática, combinando os efeitos de dispersão dos diferentes tipos de vidro. Embora as lentes acromáticas reduzam a aberração cromática, elas não a eliminam completamente, especialmente em grandes aberturas.
\end{itemize}

\textbf{Telescópios Refletores:}
\begin{itemize}
    \item \textbf{Abordagem com Espelhos Parabólicos:} Telescópios refletores utilizam espelhos parabólicos para focar a luz. A aberração cromática não é um problema nos espelhos, pois a luz é refletida, não refratada. No entanto, os espelhos precisam ser fabricados com precisão para evitar outras formas de aberração, como a aberração esférica.
\end{itemize}

\textbf{Comparação:}
\begin{itemize}
    \item \textbf{Correção de Aberração Cromática:} Enquanto os telescópios refratores corrigem a aberração cromática usando lentes acromáticas, os telescópios refletores não têm esse problema intrínseco devido ao uso de espelhos.
    \item \textbf{Desvantagens:} As lentes acromáticas ainda podem apresentar alguma aberração cromática residual, especialmente em telescópios com aberturas maiores. Por outro lado, os espelhos precisam ser fabricados com precisão para evitar outras formas de aberração.
\end{itemize}



\subsection{Aberração cromática}
A aberração cromática é um problema óptico que causa a dispersão da luz em diferentes cores, levando a imagens borradas e desfocadas. Tanto em telescópios refratores quanto refletores, a correção da aberração cromática é crucial para obter imagens nítidas e precisas. Vamos comparar como os telescópios refratores e refletores abordam esse problema e como as lentes acromáticas e os espelhos parabólicos são utilizados para corrigi-lo.


\end{document}
