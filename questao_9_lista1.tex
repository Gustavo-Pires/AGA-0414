\documentclass{letter}
\usepackage[utf8]{inputenc}
\usepackage[portuguese]{babel}
\usepackage[left=2cm, right=2cm, top=1.5cm, bottom=1.5cm]{geometry} % Ajuste das margens

\begin{document}

\begin{letter}{}

\opening{Prezados Seguradora Porto Seguro,}

Venho por meio desta carta informar sobre o sinistro ocorrido com meu veículo, conduzido por meu filho Gustavo. Inicialmente, omiti essa informação por acreditar que apenas eu poderia dirigir o veículo. Meu filho reside em São Paulo devido aos seus estudos na USP e raramente utiliza o veículo, conforme mencionado pelo corretor. Conforme explicado, tenho direito ao seguro, pois sou a principal condutora e as informações do perfil do seguro estão corretas.

Gostaria de destacar minha recusa em assinar um documento em branco solicitado pelo investigador. Tive receio de que minha assinatura fosse utilizada de maneira inadequada. Além disso, fui coagida a gravar uma conversa na qual também forneci informações falsas, com medo das possíveis consequências caso admitisse que meu filho estava dirigindo.

Após o acidente, fui ao hospital buscar o prontuário médico do condutor do veículo segurado, conforme solicitado por vocês. No entanto, fui informada de que o prontuário já havia sido entregue ao investigador da seguradora. Esta ação, feita sem nossa autorização, levanta preocupações, pois as normas do CRM e a Lei Geral de Proteção de Dados Pessoais estabelecem que apenas o paciente ou seus representantes legais podem acessar cópias do prontuário, salvo em circunstâncias excepcionais.

Quanto ao prontuario que indica "hálito etílico", é importante esclarecer que isso não implica que meu filho estivesse sob a influência de álcool. Ele apenas admitiu ter consumido uma pequena quantidade de álcool. De acordo com o Código de Trânsito, para afirmar que um condutor está sob influência de álcool, são necessários exames específicos, como o teste do bafômetro ou exame de sangue. O prontuário médico indica apenas a presença de "hálito etílico", juntamente com informações incorretas, como o diagnóstico e atendimento em clínica pediátrica. 

É crucial ressaltar que, independentemente de quem estivesse dirigindo ou se houve consumo de álcool, o acidente não poderia ter sido evitado, pois o animal atravessou a via repentinamente.

Peço desculpas pela omissão inicial, motivada pelo receio de que o seguro não cobrisse o sinistro. Como cliente fiel por muitos anos, é difícil aceitar a possibilidade de não receber suporte em momentos de extrema necessidade, como este acidente. Solicito que considerem cuidadosamente nossa situação e analisem o caso com sensibilidade.

Agradeço pela atenção dispensada e aguardo ansiosamente por uma resposta.

\closing{Atenciosamente,}

\fromname{Marta}

\end{letter}
\end{document}
